\section{Introduction}

Ce rapport a été réalisé dans le cadre du projet enjeu Santé
et Biotechnologies à CentraleSupélec.
Le but premier de ce projet est de se retrouver face
à un problème ouvert où il n'existe pas forcément de solution exacte
comme nous sommes habitués à avoir dans nos autres cours.
Un autre but poursuivi est de se retrouver en situation concrète
de collobaration avec un client extérieur,
ce qui change complètement du cadre habituel des travaux de groupes.
Finalement, ce projet nous a appris à travailler en autonomie ainsi
qu'à mener des recherches selon une démarche scientifique.

Notre projet enjeu se concentre sur
le \emph{développement d'outils mathématiques 
pour l'agriculture de précision} avec l'équipe Digiplante.
La première partie de ce titre est assez compréhensible,
la deuxième cependant suscite de nombreuses questions qui méritent
des réponses.
La croissance exponentielle de la population au cours des dernières
années a obligé la science à avancer dans le domaine de l'agriculture.
Le contexte de réchauffement climatique oblige également 
à suivre une démarche scientifique qui se veut respectueuse 
de l'environnement.


