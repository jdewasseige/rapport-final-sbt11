\section{Introduction}


Au cours de ce projet, nous nous sommes intéressés au \emph{développement d'outils mathématiques 
pour l'agriculture de précision} en collaboration avec le laboratoire MAS (Mathématiques et Informatique pour la Complexité et les Systèmes).

Dans le contexte environnemental actuel, et alors que 2 milliards de personnes sont en sous-nutrition dans le monde, la modélisation de la croissance des plantes prend tout son sens.

En effet, le problème des ressources \textit{énergétiques} et \textit{alimentaires} est un sujet crucial du  21ème siècle.
Il faudra être capable de nourrir plus de 
9 milliards d'humains en 2050~\cite{wiki:popu_mondiale}.
De plus, les ressources fossiles et l'eau douce viennent à manquer dans de nombreuses régions agricoles (comme en Californie). Cependant, l'agriculture nécessite un apport d'eau important et consomme beaucoup d'énergie. L'agriculture est ainsi responsable de l'émission de près de 20\% des gaz à effet de serre tous les ans~\cite{GES}.

Les progrès de l'informatique et des mathématiques permettent néanmoins d'espérer une amélioration nette des rendements. L'objectif final est clair : nourrir plus de monde, en minimisant l'utilisation de pesticides, d'engrais et d'eau, et donc limiter l'impact environnemental et le réchauffement climatique. En effet, les capteurs, les satellites, les drones... permettent de collecter des données de plus en plus précises. Dans un futur proche, des modèles de plus en plus efficaces permettront sans doute de tirer parti de ses données pour améliorer considérablement les rendements.  

L'objectif premier de notre projet était d'implémenter un modèle , le modèle LNAS appliqué au blé, sur la plateforme PyGMAlion-Julia du laboratoire Digiplante. Cette implémentation constituait pour ainsi dire le livrable qui était attendu par notre client Pierre Carmier.
Une fois implémenté, nous avons pu utiliser ce modèle pour simuler les caractéristiques d'un champ de blé. Par exemple, le modèle permet d'obtenir l'évolution de la biomasse des grains de blé (quantité d'intérêt) dans le champ.

Notre client Pierre Carmier, ainsi que notre référent pédagogique Paul-Henry Cournède sont des chercheurs au laboratoire MAS
Ce laboratoire travaille notamment sur les modèles mathématiques de
croissance des plantes, en collaboration avec la startup Digiplante.

A terme, nous devrions également réaliser une analyse de sensiblité qui permettrait de quantifier l'influence des paramètres qui interviennent dans le modèle. Nous avons aussi en projet de modifier quelques aspects du modèle, en rajoutant des phénomènes comme la diffusion de l'eau dans le sol...