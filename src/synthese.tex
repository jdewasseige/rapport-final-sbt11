\section{Synthèse de l'étude documentaire}
Cette partie se présente comme une synthèse de l’analyse bibliographique que nous avions réalisée en décembre.
Le lecteur intéressé trouvera une description plus détaillée des paragraphes suivants
dans les annexes~\ref{ann:physiologie},~\ref{ann:histoire},~\ref{ann:caracteristique}
et~\ref{ann:modele}.

Nous nous étions d’abord intéressés à la \emph{physiologie} de la plante. Cela nous a permis d’appréhender les éléments essentiels à son développement.
L’élément crucial qui permet le développement d’une plante est la \emph{photosynthèse}. En effet, la plante possède cette capacité extraordinaire de synthétiser de la matière organique à partir de matière minérale. 
Parce qu’ils interviennent dans la réaction de photosynthèse, l’eau, le soleil\dots ~sont des éléments clés de la modélisation du développement d’une plante.

Grâce à l’\emph{histoire} de l’étude des plantes, nous avons également pu comprendre le contexte dans lequel la modélisation des plantes s’inscrit aujourd’hui. C’est l’essor des mathématiques et de l’informatique qui permet d’envisager demain une modélisation efficace, avec des applications comme l’agriculture de précision. 
En effet, les mathématiques ont peu à peu formalisé la connaissance des plantes pour obtenir des modélisations plus précises, en introduisant des notions comme le temps thermique, le LUE et le PAR.

Nous nous étions également intéressés aux \emph{caractéristiques} propres à la modélisation des plantes, afin de comprendre les principales difficultés qui empêchent une utilisation globale des modèles de croissance des plantes. Parmi eux, on peut citer : l’estimation difficile des paramètres d’un modèle, des techniques sophistiquées en informatique et en mathématique sont nécessaires et une diversité importante sans benchmarking des modèles ne permet pas de choisir facilement celui qui sera le plus adapté. 

Nous nous sommes ensuite intéressés aux \emph{modèles les plus répandus} pour décrire le développement de la plante. Parmi eux, la grammaire L-système, le modèle GreenLab et le modèle LNAS appliqué à la betterave. La description fournie par les L-systèmes étant trop précise, ils conduisent à des temps de calcul trop importants et il est difficile d’estimer les nombreux paramètres qui interviennent.
Le modèle GreenLab introduit une alternative intéressante : plutôt que de décrire l’architecture individuelle au niveau d’un champ, les organes (les feuilles, les grains…) sont remplacés par des pools de biomasse. On ne s’intéresse plus qu’à la masse totale des feuilles, des grains, des tiges… sur un champ. Un pool de biomasse commun permet de décrire la biomasse obtenue grâce à la photosynthèse. L’enjeu est maintenant de pouvoir décrire comment la biomasse sera répartie entre les différents organes de la plante à un instant $t$. Le modèle LNAS appliqué à la betterave nous a ainsi permis de nous familiariser avec un premier modèle de manière plus concrète, grâce aux formules d’allocation de biomasse de ce modèle.
