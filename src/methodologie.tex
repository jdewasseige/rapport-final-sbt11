


\section{Méthodologies}
Nous présenterons dans cette section l'organisation générale du groupe et les outils qui nous aidé tout au long de ce projet.
On abordera d'abord les outils que nous avons utilisés pour partager 
notre travail le plus efficacement possible.
On présentera ensuite notre manière de gérer la bibliographie
ainsi que les logiciels utilisés pour ne pas perdre les sources visitées.
On détaillera ensuite l'organisation interne du groupe, et
plus précisément le système mis en place pour optimiser la communication.
On décrira finalement les problèmes et les difficultés auxquels nous avons été confrontés tout au long de ce projet.


\subsection{Outils de travail collaboratif}
Pour des raisons pratiques et esthétiques, nous avons décidé d'écrire
nos rapports en \LaTeX{}.
Il s'agissait donc de trouver la meilleure façon de partager le code source
et de pouvoir contrôler les changements apportés au document.
Une première idée pourrait être d'utiliser ShareLaTeX qui propose une plate-forme
de compilation en ligne ainsi qu'un système de gestion de versions
assez simple à utiliser.
Nous n'avons pas choisi cette solution notamment pour les raisons suivantes.
L'utilisateur doit être connecté dès qu'il veut travailler sur le projet,
le système de compilation est assez lent et l'utilisateur n'est pas libre
d'utiliser son éditeur de texte ou son visualisateur de \textsc{pdf} favori.

Pour palier aux problèmes décrits ci-dessus, le logiciel \texttt{git}
associé à GitHub est une très bonne alternative.
Il permet en effet à chaque membre du groupe de travailler sans être connecté
ainsi que d'utiliser son éditeur et compilateur favori.
Chaque membre travaille donc de son côté en faisant des \emph{commits}
et lorsqu'il juge que son travail est utile pour les autres, 
il \emph{push} sur le serveur.
L'algorithme du fusion, \emph{merge}, permet également de fusionner intelligemment
les lignes d'un fichier qui ont été modifiées par plusieurs membres.
Le dernier point à souligner est que \texttt{git} permet une gestion des branches,
particulièrement pratique lorsqu'on veut développer une partie du projet
sans risquer de créer des erreurs dans le programme principal.

Nous combinons donc ces deux outils pour 
\begin{enumerate}
  \item implémenter le modèle LNAS blé dans la plateforme
  Pygmalion en \textsc{Julia} %(une description plus détaillée de
%  l'objectif attendu pour cette partie est décrite dans la
%  section~\ref{sec:contexte} à la page~\pageref{sec:contexte})
  pour le client dont le code source est sur la plateforme GitLab,
  \item rédiger l'étude documentaire en partageant le code \LaTeX{}
  à l'aide d'un dossier sur 
  GitHub\footnote{\url{https://github.com/jdewasseige/projet-sbt11}}.
\end{enumerate}


\subsection{Gestion de la bibliographie}
Pour la gestion de la bibliographie au sein du document,
nous utilisons le package \texttt{biblatex}.
Celui-ci permet d'écrire l'ensemble de nos références dans un fichier \texttt{.bib}
sous la forme suivante.
\begin{verbatim}
  @online{histoire_mod_plantes,
    title = {Une histoire de la modélisation des plantes},
    author = {Philippe de Reffye and Marc Jaeger 
    and Paul-Henry Cournède},
    url = {https://interstices.info/jcms/c_38032/une-histoire-de-
    la-modelisation-des-plantes},
    year = {2009},
    month = "04",
  }
\end{verbatim}
La mise en page est alors automatique en fonction des informations fournies
et le rendu de l'exemple est présenté ci-dessous.
\begin{figure}[h]
  \includegraphics[scale=0.6]{./img/rendu_elem_bib.jpg}
\end{figure}

Cela parait à priori assez lourd d'écrire soi-même toutes les informations
en suivant cette syntaxe mais il existe des logiciels comme Zotero
qui font le travail à notre place.
Les sources trouvées sur Google Scholar peuvent également être exportées
aisément au format \texttt{BibTex}.

\subsection{Organisation et partage des tâches}
Afin de communiquer et de prévoir notre travail, nous avons essentiellement utilisé Slack\footnote{\url{https://slack.com/}}, qui est un logiciel
de plus en plus utilisé pour les travaux de groupe ainsi que dans les start-ups.
Il permet d'éviter de devoir alterner entre plusieurs applications comme les mails,
DropBox et Twitter, puisqu'il permet d'être connecté
à celles-ci au sein de l'application.
On peut également créer plusieurs \emph{channels} pour séparer la communication
entre les différentes tâches.
Par exemple dans ce projet nous avons les \emph{channels} suivantes :
\texttt{general}, \texttt{etude-documentaire}, 
\texttt{planning} et \texttt{dev\_plate-forme}.
On trouve aussi un système d'historique et de gestion de fichiers efficace.

\subsection{Planning}
L'organisation de nos travaux au cours de ce semestre
sont présentés en annexe~\ref{ann:planning}
dans le tableau~\ref{fig:planning}
(page~\pageref{fig:planning}).
Il contient ce qui devait être fait en théorie et à quel moment,
certaines semaines voyaient leurs tâches réalisées entièrement
tandis que d'autre non.
Nous présentons donc pour la dernière partie de cette section
les différentes complications rencontrées.

\subsection{Difficultés rencontrées}

Au cours de ce projet, nous avons été confrontés à de nombreuses difficultés.

\subsubsection{Difficultés de gestion du temps}

Le temps fut une difficulté récurrente. Nous n'étions sans doute pas prêts à gérer de nous-mêmes notre temps. C'est ainsi que nous avons réalisé l'opportunité que constitue ce type de projet. 
Jusqu'à maintenant, nous avions été confrontés la plupart du temps à des questions précises dans des examens, dont l'emploi du temps nous est imposé à l'avance. 
Ici, nous devions de nous même organiser notre temps. 
Nous pourrions nous dédouaner en soulignant l'emploi du temps chargé de nos études à Centrale. 
Mais ce projet est là pour nous rappeler qu'une organisation préalable et une répartition efficace des tâches doivent permettre d'éviter de subir le temps. Et cela s'apprend grâce à des projets comme celui-ci.

\subsubsection{Difficultés informatiques}

Nous avons également été confrontés à des problèmes informatiques lors de l'implémentation et de l'utilisation du modèle.
\begin{itemize}
	\item Un bogue a empêché au début de nos premières tentatives le lancement de nos programmes. En effet, les chercheurs
du laboratoire MAS qui ont développé la plateforme utilisait les fonctions directement sur leur répertoire, alors que nous y accédions depuis le dossier principal. Il y avait un problème de chemin relatif. Il a donc fallu détecter l'origine de ce bogue avant de pouvoir commencer à utiliser nos programmes. 
	
	\item De plus, des erreurs classiques sur notre implémentation empêchaient également le bon déroulement de nos utilisations du modèle : coquilles, oublis, fautes de frappe... L'impossibilité d'utiliser la simulation à cause du bogue précédemment décrit a rendu leur détection plus ardue.

\end{itemize}
\subsubsection{Difficultés théoriques}
Enfin, le modèle directement implémenté à l'aide du document qui contenait les formules théoriques du modèle, et que nous avons implémenté, conduisait à des valeurs qui n'étaient pas cohérentes. Les connaissances de Pierre Carmier sur les modèles de croissance des plantes nous ont permis de savoir ce qui était cohérent dans nos résultats et ce qui ne l'était pas. 
Nous avons donc avec l'aide de notre client ajusté les formules que nous utilisions :  
\begin{itemize}
	\item Dans un premier temps, la croissance de la plante était trop faible. Il a fallu changer les fonctions log-normales dans les fonctions d'allocation, de remobilisation et de sénescence. 
La formule donnée dans le document  utilisait une définition différente de l'écart type que la formule qui correspondait aux paramètres que Pierre Carmier nous avait fournis. Nous avons donc utilisé directement la formule déjà implémentée dans le modèle LNAS betterave qui correspond bien aux conventions utilisées.
	\item Une fois ceci corrigé, nous obtenions cette fois des récoltes beaucoup trop élevées, plus de 3000 en grains et environ 16 de LAI, alors que les valeurs attendues sont respectivement en ordre de grandeur 1000 et entre 5 et 6.
Cela était dû à l'absence de prise en compte dans le modèle théorique du temps de montaison, temps à partir duquel la tige commence à croître de manière très rapide.

En effet, nous procédions comme suit pour l'allocation de la biomasse produite par photosynthèse aux différents organes : le coefficient d'allocation des grains d'abord calculé. Il est nul au départ et tend rapidement vers 1 lorsque la plante arrive à maturité. La biomasse restante était ensuite allouée à parts égales entre les 3 compartiments restants (tiges, racines, feuilles) (sauf en cas de stress thermique ou hydrique). Ceci est peu réaliste.
Nous avons donc introduit un temps de montaison, tout d'abord de manière assez abrupte. 
Il s'agit de ne rien allouer à la tige au départ, puis de lui en allouer de plus en plus à partir du temps de montaison. Nous nous sommes contentés en première approximation de décrire deux régimes, un régime pré-montaison et un régime post-montaison, avec une transition instantanée. Plus tard, nous comptons implémenter une transition progressive avec une loi lognormale et rentrer ces nouveaux paramètres dans le modèle.
Cela a permis d'avoir des valeurs plus réalistes pour le LAI et la quantité de grain est descendue à 2500.

	\item Mais nous obtenions toujours une trop grande quantité de grain (2500). 

Cela était dû au fait qu'après la maturité de la plante, toute la biomasse de la tige était réallouée aux grains, ce qui n'est pas réaliste non plus. En effet, une partie des tiges subit une sénescence trop avant de pouvoir se transformer en grains de blé. On crée donc un compartiment ``tige jaune'', suivant le modèle qu'on a feuille jaune/feuille verte. La biomasse de ce compartiment ne sera jamais allouée aux grains.
Après ajout de ce compartiment au modèle, nous avons obtenu des valeurs de grain finales légèrement en dessous de 2000, ce qui était encore trop. 
Nous regarderons après la soutenance si le modèle peut encore être affiné, mais surtout nous essaierons de jouer sur les paramètres du modèle pour obtenir des valeurs encore plus cohérentes.


\end{itemize}

