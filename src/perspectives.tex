\subsection{Perspectives}

Notre objectif premier, qui était d'implémenter le modèle LNAS appliqué au blé sur la plateforme Pygmalion-Julia, il s'agira après la soutenance de continuer le projet en affinant le modèle, en l'étudiant et en l'utilisant sur un jeu de données expérimentales pour comparer les simulations du modèle aux données réelles.

Ainsi, nous envisageons par exemple de décrire de manière plus précise l'eau dans le sol : pour l'instant, celle-ci est répartie équitablement dans tout le domaine. Nous pourrions introduire dans le modèle la capillarité de l'eau, qui a tendance à la faire remonter, et la gravité, qui l'entraîne vers le bas.

Pierre Carmier nous enjoint également à réaliser une analyse de sensibilité, pour déterminer l'influence