\subsection{Perspectives}

Notre objectif premier, qui était d'implémenter le modèle LNAS appliqué au blé sur la plateforme Pygmalion-Julia, étant atteint, il s'agira après la soutenance de continuer le projet en affinant le modèle, en l'étudiant et en l'utilisant sur un jeu de données expérimentales pour comparer les simulations du modèle aux données réelles.

Ainsi, nous envisageons par exemple de décrire de manière plus précise le comportement de l'eau dans le sol. Ainsi, nous pourrions introduire dans le modèle la capillarité de l'eau, qui a tendance à la faire remonter des couches les plus humides aux couches les plus sèches, et décrire plus précisément le drainage de l'eau vers le bas dû tout simplement à la pesanteur. Il s'agira de comparer le comportement du modèle avec et sans ce raffinement, afin de constater ou non son utilité.

Pierre Carmier nous enjoint également à réaliser une analyse de sensibilité, pour déterminer l'influence de chaque paramètre sur le modèle. En effet, plus de 20 paramètres sont présents dans le modèle, et les déterminer tous nécessite de nombreux jeux de données. Si l'influence de certains paramètres s'avère assez faible, on pourra les remplacer par des constantes prises à leur valeur moyenne estimée pour simplifier l'étape de l'estimation des paramètres.

Nous aimerions également utiliser le modèle sur un jeu de données expérimentales. L'utilisation du modèle se décrit comme suit : on rentre un fichier de données expérimentales et on utilise une méthode d'estimation des paramètres (comme la méthode des moindres carrés généralisés par exemple, qui est déjà implémentée sur la plateforme). Cela nous permet d'obtenir une estimation des paramètres. 
Une fois les paramètres estimés, on utilise ceux-ci pour à nouveau réaliser des simulations avec 
le modèle. En effet, cette fois les paramètres sont fixés à leur valeur estimée et on ne rentre que les conditions initiales et extérieures du champ considéré.
Avec un autre jeu de donnée, on peut ainsi confronter les simulations obtenues avec les valeurs expérimentales. On vérifie ainsi la bonne adéquation du modèle, the goodness of fit. On peut ainsi obtenir le niveau d'incertitude de la prédiction fournie par le modèle.
Si cette étape s'avère concluante, on peut alors réaliser des prédictions sur à priori n'importe quel champ de blé. Ce sont les étapes logiques de modélisation. 

