\subsection{Travail à venir}
Notre objectif premier, qui était d'implémenter le modèle LNAS appliqué au blé sur la plateforme Pygmalion-Julia, étant atteint, il s'agira après la soutenance de continuer le projet en affinant le modèle, en l'étudiant et en l'utilisant sur un jeu de données expérimentales pour comparer les simulations du modèle aux données réelles.

Nous avons bien avancé et suivi la méthodologie de bonne pratique en modélisation. \ref{ann:caracteristique}

Nous avons terminé la première étape d'analyse et étudié les comportements généraux du modèle.
Pierre Carmier nous enjoint également à réaliser une analyse de sensibilité, pour déterminer l'influence de chaque paramètre sur le modèle. En effet, plus de 20 paramètres sont présents dans le modèle, et les déterminer tous nécessite de nombreux jeux de données. Si l'influence de certains paramètres s'avère assez faible, on pourra les remplacer par des constantes prises à leur valeur moyenne pour simplifier le modèle et rendre les estimations plus faciles.

Ensuite il nous faudra identifier le modèle. C'est-à-dire en faire une estimation non-paramétrique (structure du modèle, choix de modélisations...) et paramétriques (valeur des paramètres) en le confronter à un jeu de données expérimentale. L'estimation des paramètres du modèle se décrit comme suit : On rentre un fichier de données expérimentales et on utilise une méthode d'estimation des paramètres (comme la méthode des moindre carrés généralisés par exemple, qui est déjà implémentée sur la plateforme). Cela nous permet d'obtenir une estimation des paramètres. Des modifications qualitatives au modèles peuvent être fait à la main ou alors mises sous formes de paramètres booléens (par exemple, "use advanced water simulation = true or false").

Enfin il s'agirait d'évaluer le modèle. Une fois les paramètres estimés, on peu ensuite utiliser le modèle pour réaliser des simulations et les confronter à des données expérimentales qui n'ont pas servi à déterminer les paramètre pour voir si le comportement qualitatif et quantitatif est satisfaisant : évolutions cohérentes, goodness of fit des données, incertitudes.
Si cette étape s'avère concluante, on peut alors tester les capacités prédictives du modèles et voir en voir ainsi des applications pratiques.

La modélisation n'est pas linéaire et il s'agit de faire des allers-retours entre ces étapes. Par exemple, pour affiner ce modèle, on pourrait décrire de manière plus précise le comportement de l'eau dans le sol. Ainsi, nous pourrions introduire dans le modèle la capillarité de l'eau, qui a tendance à la faire remonter des couches les plus humides aux couches les plus sèches, et décrire plus précisément le drainage de l'eau vers le bas dû tout simplement à la pesanteur. Il s'agira de comparer le comportement du modèle avec et sans ce raffinement, afin de constater ou non son utilité, s'il donne des résultats plus réalistes ou pas. On peut l'introduire en paramètre booléen comme suggéré plus haut.

Ainsi le temps qui nous reste sera consacré au travail pratique sur ce modèle.
