\section*{Résumé}
\addcontentsline{toc}{section}{Résumé}
Ce rapport présente nos résultats et nos perspectives
du projet et s'articule autour de cinq parties.

Dans un premier temps, nous introduisons le contexte global du
projet en mettant l'accent sur l'objectif poursuivi par celui-ci.

On expose ensuite une synthèse de l'\emph{étude documentaire}
en décrivant brièvement chacun des thèmes abordés dans celle-ci.
On rappelera notamment quelques notions de physiologie des plantes, un bref historique de la modélisation des plantes
ainsi que les principaux modèles actuels et leurs caractéristiques communes.

La partie suivante décrit l'ensemble des méthodes
utilisées pour mener à bien notre projet.
On décrira d'abord les outils de travail que nous avons
utilisés afin de collaborer efficacement.
On rappelera ensuite la façon dont nous avons
géré notre bibliographie.
Un planning de la répartition des tâches à travers
le semestre sera ensuite donné.
On terminera cette partie en décrivant de façon
exhaustive les difficultés que nous avons rencontrées
jusqu'à maintenant.

Avant de décrire les résultats obtenus,
nous rappelerons de manière synthétique le fonctionnement
de notre modèle.
On expliquera ensuite comment nous avons implémenté
celui-ci en \textsc{Julia} en détaillant les
variables et fonctions utilisées.

La dernière partie reprendra d'une part nos résultats
ainsi qu'une analyse de ceux-ci et de leur pertinence
par rapport à ce que l'on observe expérimentalement,
et d'autre part notre plan de travail pour la fin 
de l'année et les perspectives à long terme
que nous envisageons pour le projet.
