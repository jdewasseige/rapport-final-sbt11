\section{Résumé}

Dans un premier temps, nous exposons une \emph{synthèse documentaire}
balayant le plus largement possible le sujet de l'agriculture de précision,
synthèse qui est le fruit de notre étude bibliographique.
Les références utilisées se trouvent en annexe.
Au cours de cette synthèse, nous commençons par poser les bases
nécessaires, en introduisant le concept
d'agriculture de précision, des notions de physiologie des plantes
ainsi qu'un bref historique de la compréhension des plantes.
Nous entrons ensuite dans le vif du sujet en détaillant quelques modèles
et leurs caractéristiques mathématiques.

On retrouve ensuite une partie sur le contexte
et les objectifs détaillés du projet.
Les différents outils qui nous ont aidés dans la réalisation de
cette partie se trouvent en annexe.

Nous finirons par expliquer l'organisation interne du groupe
ainsi que les moyens mis en \oe{}uvre pour assurer
le bon déroulement du projet.