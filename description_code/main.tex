\section{Description du fonctionnement du code}
Dans un premier temps, nous avons construit un modèle informatique à partir
du modèle mathématique fourni~\cite{lnas_model_wheat}.

Il s'agissait d'abord d'assigner à chaque variable du modèle un représentant
dans notre programme, en rendant ces derniers assez explicites
afin d'être plus efficace pour l'implémentation des fonctions.
Les tableaux~\ref{table:state_var},~\ref{table:control_var} et~\ref{table:param_var}
contiennent respectivement les variables d'\emph{état} auxiliaires et principales, 
les variables \emph{environnementales} et les \emph{paramètres} de la tige, du sol et des racines.


\begin{table}[H]
  \centering
  \begin{tabular}{|c|c|}
    \hline
    \textbf{Modèle mathématique} & \textbf{Implémentation} \\
    \hline
    $Q_r$ & root\_biomass \\
    $Q_s$ & stem\_biomass \\
    $Q_l$ & green\_leaf\_biomass \\
    $Q_g$ & grain\_biomass \\
    $Q_y$ & yellow\_leaf\_biomass \\
    $\theta$ & soil\_humidity \\
    $\tau$ & thermal\_time \\
    $R$ & soil\_contained\_water \\
    $E_s$ & soil\_water\_evaporated \\
    $T_p$ & water\_transpired \\
    $z_r$ & root\_horizon \\
    SSI & stomatal\_stress\_index \\
    TSI & thermal\_stress\_index \\
    TSI$_{\uparrow}$ & thermal\_stress\_index \\
    TSI$_{\downarrow}$ & soil\_thermal\_stress\_index \\
    % Auxiliary
    $q$ & produced\_biomass \\
    LAI & leaf\_area\_index \\
    E\textsubscript{spot} & soil\_req\_evaporation \\
    T\textsubscript{ppot} & req\_transpiration \\
    E\textsubscript{smax} & soil\_max\_evaporation \\
    T\textsubscript{pmax} & max\_transpiration \\
    T$_{\downarrow}$ & soil\_temperature \\
    % Temporary
    $\alpha_g$ & alpha\_g \\
    $\alpha_s$ & alpha\_s \\
    $\alpha_r$ & alpha\_r \\
    $\alpha_l$ & alpha\_l \\
    $\beta_s$ & beta\_s \\
    $\beta_l$ & beta\_l \\
    $\gamma_l$ & gamma\_l \\
    $\gamma_y$ & gamma\_y \\
    \hline
    \end{tabular}
  \caption{Variables d'\emph{état} du modèle mathématique et leurs noms
  dans l'implémentation.}
  \label{table:state_var}
\end{table}

\begin{table}[H]
  \centering
  \begin{tabular}{|c|c|}
    \hline
    \textbf{Modèle mathématique} & \textbf{Implémentation} \\
    \hline
    $T$ & vec\_ext\_temperature \\
    PAR & vec\_par \\
    $W$ & vec\_water\_input \\
    ET0 & vec\_et0 \\
    \hline
    \end{tabular}
  \caption{Variables de \emph{contrôle} du modèle mathématique et leurs noms
  dans l'implémentation.}
  \label{table:control_var}
\end{table}

\begin{table}[H]
  \centering
  \begin{tabular}{|c|c|}
    \hline
    \textbf{Modèle mathématique} & \textbf{Implémentation} \\
    \hline
    $t_c$ & t\_c \\
    $t_{opt}$ & t\_opt \\
    $\mu_g$ & mu\_g \\
    $\sigma_g$ & sigma\_g \\
    $\eta_s$ & eta\_s \\
    $\eta_l$ & eta\_l \\
    $\tau_l$ & tau\_l \\
    $\mu_l$ & mu\_l \\
    $\sigma_l$ & sigma\_l \\
    $\tau_y$ & tau\_y \\
    $\mu_y$ & mu\_y \\
    $\sigma_y$ & sigma\_y \\
    RUE & rue \\
    $\lambda$ & lambda \\
    $\rho_l$ & rho\_l \\
    $K_c$ & k\_c \\
    $K_s$ & k\_s \\
    $\theta_{max}$ & theta\_max \\
    $\theta_{min}$ & theta\_min \\
    $z_s$ & z\_s \\
    $z_m$ & z\_m \\
    $\rho_r$ & rho\_r \\
    $t_{\downarrow c}$ & t\_soil\_c \\
    $t_{\downarrow opt}$ & t\_soil\_opt \\
    \hline
  \end{tabular}
  \caption{Paramètres du modèle mathématique et leurs noms
  dans l'implémentation.}
  \label{table:param_var}
\end{table}
  






La deuxième partie du travail consistait à définir les différentes fonctions
qui mettent à jour les variables d'état et paramètres du 
temps $n$ au temps $n+1$.
L'implémentataion de chacune d'entre elle est basée sur l'algorithme générique~\ref{alg1}.
On retrouve à l'entrée les variables d'états \texttt{xn} et \texttt{xnplus1},
le temps \texttt{n}, les variables de contrôles \texttt{u}, les paramètres \texttt{p}
et éventuellement des informations supplémentaires qui dépendent
du rôle précis de la fonction.
On va ensuite mettre à jour une composante de \lstinline{xnplus1}
en suivant la description du modèle.

\begin{algorithm}[caption={Algorithme générique qui sert de base pour l'implémentation
  des fonctions. La fonction $f$ n'est pas définie mais sert de placeholder
  pour représenter les opérations nécessaires à la mise à jour de xnplus1.}, label={genfun}]
 input: int n, State Vector xn, Control Vector u, Parameters Vector p, State Vector xnplus1
 output: None
 begin
   xnplus1.composante $\gets$ f(p.composante, u.composante, xn.composante)
 end       
\end{algorithm}

À titre d'exemple, on va utiliser la fonction \texttt{get\_pot\_evaporation} qui met
à jour l'évaporation requise en fonction des conditions environnementales
selon l'équation~\ref{eq:req_evap}.
\begin{equation}
  \text{Espot}^{(n)} = K_s \text{ ET0 } \exp{-\lambda \text{ LAI}^{(n)}}
  \label{eq:req_evap}
\end{equation}
En suivant la démarche~\ref{genfun}, on obtient en \textsc{Julia}
l'implémentation suivante.
\lstlisting{
function get\_pot\_evaporation!(n, xn, u, p, xnplus1)
	xnplus1.soil\_req\_evaporation = p.k\_s * u.ET0[n] * 
  exp( -p.lambda * xn.leaf\_area\_index) 
end
}


% On retrouvera finalement trois fonctions ne suivant pas l'algorithme générique
% \begin{enumerate}
%   \item \texttt{initialize} qui va initialiser les différentes composantes de x0,
%   \item \text{transition} qui va créer le vecteur xnplus1 et exécuter toutes les fonctions,
%   \item \text{observer\_function} [???].
% \end{enumerate}


