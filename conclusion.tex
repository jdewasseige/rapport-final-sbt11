\section*{Conclusion}
\addcontentsline{toc}{section}{Conclusion}

En conclusion, l'objectif principal qui avait été fixé a été atteint : implémenter le modèle LNAS blé sur la plateforme Pygmalion-Julia. Nous avons pu obtenir des premières simulations satfisfaisantes à partir du jeu de paramètres que nous a fourni notre client. En effet, d'un point de vue qualitatif, les courbes semblent cohérentes. 

Quelques données ne correspondent pas encore tout à fait à ce qui était attendu, c'est pourquoi nous essaierons après la soutenance d'affiner au mieux le modèle, afin d'avoir les résultats les plus cohérents possibles

Une prochaine étape intéressante sera également d'utiliser le modèle sur un jeu de mesures expérimentales pour confronter notre modèle à la réalité expérimentale.

Si c'était à refaire, nous essaierions d'aller plus vite sur l'implémentation du modèle, pour pouvoir commencer le plus tôt possible la partie la plus intéressante du modèle, à savoir l'utilisation du modèle, amélioration du modèle par essai/erreur...

