\section*{Conclusion}
\addcontentsline{toc}{section}{Conclusion}

Alors que l'intelligence artificielle vient de battre le meilleur joueur du monde de go, plus rien ne semble hors de portée de l'informatique et des mathématiques, reste à appliquer cette révolution à un secteur vieux comme le monde : l'agriculture.

Pour la modélisation de la croissance des plantes, des modèles comme le modèle LNAS permettent d'envisager une simulation efficace et précise des champs de culture et à terme d'augmenter considérablement les rendements en suivant les recommandations de tels modèles. En effet, l'ordinateur pourra tester des milliers de possibilité (culture de blé, d'orge sur tel parcelle, avec plus d'eau sur le nord de la parcelle que sur le reste par exemple) et fournir celle qui donne les meilleurs rendements. 
Les enjeux sont cruciaux. Les ressources viennent à manquer, alors que l'humanité n'a jamais eu autant besoin de cultiver pour se nourrir, pour produire des bio-carburants\dots ~Une agriculture intelligente, portée par la modélisation de la croissance des plantes apparaît donc comme une bonne alternative à l'agriculture intensive.


Si nous devions conseiller une équipe qui prendrait notre suite, nous lui dirions d'aller plus vite sur l'implémentation du modèle, pour pouvoir commencer le plus tôt possible la partie la plus intéressante, à savoir l'utilisation du modèle, sa manipulation sur des données réelles et les modifications par essai/erreur. Ce modèle peut être affiné et amélioré, et testé sur de nombreuses données, ou alors d'autres modèles peuvent être implémentés. En tout cas les possibilités ne manquent pas et les pistes intéressantes sont nombreuses.

